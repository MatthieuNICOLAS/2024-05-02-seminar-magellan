\section{Synchronisation Models}

\begin{frame}{Synchronisation Models}
    \metroset{block=transparent}
    \begin{block}{To converge}
        \begin{itemize}
            \item Nodes have to propagate changes \dots
            \item \dots And integrate those of others
        \end{itemize}
    \end{block}
    \pause
    \begin{block}{Several approaches proposed \cite{shapiro_2011_crdt}}
        \begin{itemize}
            \item State-based synchronisation
            \item Operation-based synchronisation
            \item<3->
                \only<3> {Delta-based synchronisation \cite{Almeida_2018}}
                \only<4> {\color{gray} Delta-based synchronisation \cite{Almeida_2018}}
            \only<3-4>{
                \begin{itemize}
                    \item
                        \only<3>{"Best of the two worlds" approach}
                        \only<4>{\color{gray} "Best of the two worlds" approach}
                \end{itemize}
            }
        \end{itemize}
    \end{block}
\end{frame}

\subsection{State-based CRDTs}

\begin{frame}{State-based synchronisation}
    \metroset{block=transparent}

    \begin{itemize}
        \item \alert{Send periodically current state} to other nodes
    \end{itemize}

    \begin{figure}[!ht]

        \centering
        \resizebox{\columnwidth}{!}{
          \begin{tikzpicture}
            \path
                node {\textbf{A}}
                ++(0:0.5) node (a) {}
                +(0:13.5) node (a-end) {}
                +(0:1) node[point, label=above right:{$\set{a,d}$}] (a-initial) {}
                +(0:3.5) node[point, label=above right:{$\set{a,b,d}$}, label=-170:{$\trm{add}(b)$}] (a-add-b) {}
                +(0:6) node[point, label=above right:{$\set{a,b,c,d}$}, label=-170:{$\trm{add}(c)$}] (a-add-c) {}
                +(0:8.5) node (a-sends-state) {}
                +(0:11.5) node (a-final) {};

            \draw[dotted] (a) -- (a-initial) (a-final) -- (a-end);
            \draw[->, thick] (a-initial) -- (a-final);

            \path
                ++(270:2) node {\textbf{B}}
                ++(0:0.5) node (b) {}
                +(0:13.5) node (b-end) {}
                +(0:1) node[point, label=below right:{$\set{a,d}$}] (b-initial) {}
                +(0:3.5) node[point, label=below right:{$\set{\cancel{a},d}$}, label=170:{$\trm{rmv}(a)$}] (b-rmv-a) {}
                +(0:8.5) node (b-sends-state) {}
                +(0:11.5) node (b-final) {};

            \draw[dotted] (b) -- (b-initial) (b-final) -- (b-end);
            \draw[->, thick] (b-initial) -- (b-final);

            \path<2->
                (a-sends-state) node[point, label=below left:{$\trm{sync}$}, label={[xshift=10pt]-10:{$\set{a,b,c,d}$}}] {}
                (b-final) node[point] {};
            \draw<2->[->, dashed, shorten >= 1] (a-sends-state) -- (b-final);

            \path<3->
                (b-final) node[label=below right:{$\set{\cancel{a},b,c,d}$}] {};

            \path<4->
                (b-sends-state) node[point, label=170:{$\trm{sync}$}, label={[xshift=15pt]10:{$\set{\cancel{a},d}$}}] {}
                (a-final) node[point, label=above right:{$\set{\cancel{a},b,c,d}$}] {};

            \draw<4->[->, dashed, shorten >= 1] (b-sends-state) -- (a-final);
          \end{tikzpicture}
        }
    \end{figure}

    \begin{itemize}
        \item<3-> Upon reception, \alert{computes new state by merging received state with current one} using \texttt{merge} function  ($\trm{join}$ in lattice theory)
        \item<5-> With \texttt{merge}, a \alert{commutative, associative and idempotent function}
    \end{itemize}
\end{frame}

\begin{frame}{State-based synchronisation - Review}
    \metroset{block=transparent}

    \begin{block}{Strengths}
        \begin{itemize}
            \item No assumptions on the network reliability
            \item i.e. messages may be lost, re-ordered or duplicated w/o impact
        \end{itemize}
    \end{block}
    \pause
    \begin{block}{Limits}
        \begin{itemize}
            \item States and corresponding \texttt{merge} difficult to design
            \begin{itemize}
                \item e.g. how to represent efficiently deletion of elements?
            \end{itemize}
            \pause
            \item Depending on data type, states expensive to broadcast\dots
            \item \dots And \texttt{merge} expensive to execute
        \end{itemize}
    \end{block}
\end{frame}

\subsection{Operation-based CRDTs}

\begin{frame}{Operation-based synchronisation}
    \begin{itemize}
        \item \alert{Encode updates as arbitrary messages}, called \emph{operations}
        \item An operation \alert{corresponds to one or several \emph{irreducible elements}}
    \end{itemize}

    \begin{figure}[!ht]

        \centering
        \resizebox{\columnwidth}{!}{
          \begin{tikzpicture}
            \path
                node {\textbf{A}}
                ++(0:0.5) node (a) {}
                +(0:14.5) node (a-end) {}
                +(0:1) node[point, label=above right:{$\set{a,d}$}] (a-initial) {}
                +(0:3.5) node[point, label=above right:{$\set{a,b,d}$}, label=-170:{$\trm{add}(b)$}] (a-add-b) {}
                +(0:7) node[point, label=above right:{$\set{a,b,c,d}$}, label=-170:{$\trm{add}(c)$}] (a-add-c) {}
                +(0:12.5) node (a-receives-rmv-a) {};

            \draw[dotted] (a) -- (a-initial) (a-receives-rmv-a) -- (a-end);
            \draw[->, thick] (a-initial) -- (a-receives-rmv-a);

            \path
                ++(270:3) node {\textbf{B}}
                ++(0:0.5) node (b) {}
                +(0:14.5) node (b-end) {}
                +(0:1) node[point, label=below right:{$\set{a,d}$}] (b-initial) {}
                +(0:6) node[point, label=below right:{$\set{\cancel{a},d}$}, label=170:{$\trm{rmv}(a)$}] (b-rmv-a) {}
                +(0:9) node (b-receives-add-b) {}
                +(0:12.5) node (b-receives-add-c) {};

            \draw[dotted] (b) -- (b-initial) (b-receives-add-c) -- (b-end);
            \draw[->, thick] (b-initial) -- (b-receives-add-c);

            \path<2->
                (b-receives-add-b) node[point, label=below right:{$\set{\cancel{a},b,d}$}]  {}
                (b-receives-add-c) node[point, label=below right:{$\set{\cancel{a},b,c,d}$}] {};

            \draw<2->[->, dashed, shorten >= 1] (a-add-b) -- (b-receives-add-b);
            \draw<2->[->, dashed, shorten >= 1] (a-add-c) -- (b-receives-add-c);

            \path<3->
                (a-receives-rmv-a) node[point, label=above right:{$\set{\cancel{a},b,c,d}$}] {};

            \draw<3->[->, dashed, shorten >= 1] (b-rmv-a) -- (a-receives-rmv-a);
          \end{tikzpicture}
        }
    \end{figure}

    \pause

    \begin{itemize}
        \item<2-> Upon reception, \alert{apply operations on current state}
        \item<4-> \alert{Concurrent operations} must be \alert{commutative}
    \end{itemize}
\end{frame}

\begin{frame}{Operation-based synchronisation - Review}
    \metroset{block=transparent}

    \begin{block}{Strengths}
        \begin{itemize}
            \item Designing operations is straightforward
            \item Operations usually cheap to broadcast and apply
        \end{itemize}
    \end{block}

    \pause

    \begin{block}{Limits}
        \begin{itemize}
            \item Hides/delegates complexity to delivery of operations
            \begin{itemize}
                \item i.e. requires specific delivery order of operations
                \item e.g. insertion of an element before its deletion
            \end{itemize}
            \pause
            \item Weak to network failures
            \pause
            \item Have to pair Op-based CRDTs to a message delivery service
            \begin{itemize}
                \item To re-order and/or de-duplicate operations
                \item To retrieve lost operations using anti-entropy mechanisms
            \end{itemize}
        \end{itemize}
    \end{block}
\end{frame}